\documentclass[13pt]{beamer}
\usepackage{hyperref}
\usepackage{amsmath}
\usepackage{enumitem}
\usepackage{lipsum}

\usetheme{CambridgeUS}
\usecolortheme{dolphin}
%\usetheme{Malmoe}
%\usecolortheme{seahorse}

%\begin{frame}{Table of Content}
%\tableofcontents[currentsection,currentsubsection]
%\end{frame}

% redefine default beamer item labels
\setitemize{label=\usebeamerfont*{itemize item}%
  \usebeamercolor[fg]{itemize item}
  \usebeamertemplate{itemize item}}
  
    
    
\title{E-Learning Platform Development \\ Bodhitree}
\author{Pankaj S Gaikwad \\ 133059010 }

\begin{document}
\maketitle

\section{Outline}
\begin{frame}
    \frametitle{Outline}
    \begin{itemize}
	\item Introduction
	\item MOOCs variants
	\item Design perspectives
	\item Evaluation techniques
	\item Instructional quality of MOOCs
	\item MOOCs potentials
	\item Bodhitree
	\item Conclusion \& future work
    \end{itemize}
\end{frame}



\section{Introduction}
\begin{frame}
	\frametitle{Introduction}

	\begin{itemize}
	  \item E-Learning concept
	  \item Definition
	   \begin{itemize}
		 \item Massive: many participants (hundreds to thousands)
		 \item Open: free of cost, anyone can register
		 \item Online: available on internet
		 \item Courses: curriculum unit
	  \end{itemize}
	  \item Recent developments
	  \begin{itemize}
		 \item CCK'08 - First MOOC
		 \item Rise of providers: Coursera, Udacity, EdX
		 \item Other platforms: NPTEL, Open2Study
		 \item Quasi MOOCs: Khan Academy, Peer-to-peer University (P2PU)
	  \end{itemize}
	\end{itemize}
\end{frame}

\begin{frame}
	\frametitle{Challenges in MOOCs}
	    \begin{itemize}
	     \item High dropout rate
	     \item Assessment
	     \item Feedback
	     \item Lack of human interaction
	     \item Self-learning centered
	     \item Actual access to only privileged ones
	    \end{itemize}

\end{frame}


\begin{frame}
    \frametitle{MOOC vs Learning Management System}
    \begin{table}[ht]
    \begin{tabular}{|l|l|l|}
        \hline
        ~                & \textbf{MOOC Platforms}                & \textbf{LMS}                          \\ \hline \hline
        Goal             & \textit{replace} classrooms 	 & \textit{assist} classrooms \\ \hline
        Course building  & can be built   		 & Resource holder  \\ \hline
        Networking       & More (blogs)    	         & Less (wiki,forums)  \\ \hline
        Size,scalability & Large size, highly scalable   & Small size                   \\ \hline
        Access           & Lifelong access               & Limited duration access      \\ \hline
        Establishments   & Late 2000's                   & Early 1960's                 \\ \hline
        Examples         & Coursera, EdX                 & Moodle, Blackboard           \\
        \hline
    \end{tabular}
    %\caption{MOOC Platforms vs. LMS}
    \label{tab:moocvslms}
\end{table}

\end{frame}

\section{MOOCs variants}
\begin{frame}
	\frametitle{MOOC variants}
	\begin{itemize}
	  \item Attributes of MOOCs
	  \begin{itemize}
		\item Massive, open, online, courses
		\item Discussion boards
		\item Online quizzes, instruction videos, reading materials
		\item Evaluation: Peer grading and auto grading
		\item Social networking 
	  \end{itemize}
	  %\item xMOOC vs cMOOC 
	  %\item Blended MOOCs
	\end{itemize}
\end{frame}

\begin{frame}
	    \frametitle{xMOOC vs cMOOC}
%	    \begin{itemize}
%		\item Concept
%		\item Dedicated platform
%		\item Content orientation
%		\item Assignments/Exams
%		\item Feedback
%		\item Learning analytics
%		\item Certification
%	    \end{itemize}
\begin{table}[ht]
\resizebox{\textwidth}{!}{
    \begin{tabular}{|l|l|l|}
        \hline
        Criteria/Variant           & \textbf{xMOOC}                      & \textbf{cMOOC}                                                                        \\ \hline \hline
        Full form                   &  Extensive MOOCs                     & Connectivist MOOCs                                                           \\ \hline
        Dedicated platform 	   & Yes                                 & No, (shared platform/ \\
        ~			   &	~				 &social media)                                    \\ \hline
        Content                    & Instructor driven   		 & Participant driven                                                    \\ \hline
        Network Dependence         & Large, few active participants 	 & Moderate, active learners                                                    \\ \hline
        personal interactions	   & Very less 				 & High                                                    \\ \hline
        Assignments/Exams          & Computer based online               & No formal assignments                                                        \\ \hline
        Discussion Space           & Dedicated, unmoderated              & Use of social \\
        ~			   & ~					 & networking sites \\ \hline
        Assessment                 & Auto-grading                        & Peer grading            \\ \hline
        Feedback                   & Not supported                       & Peer feedback     \\ \hline
        Learning Analytics         & Supported                           & Not supported                                                                \\ \hline
        Certification	           & Supported 				 & Not supported \\
        ~ 			   & not accepted (non-credit)		 &(autonomous learners)                                         \\ \hline
        Examples		   & Coursera, EdX			 & EduMOOC, CCK08 \\
        \hline
    \end{tabular}}
    \caption{xMOOC vs. cMOOC}\label{tab:xmoocvscmooc}
\end{table}
\end{frame}

\begin{frame}
	    \frametitle{Adaptive MOOCs}
	    \begin{itemize}
		\item Concept
		\item Learning strategies: apprentice, inductive, incidental, deductive, discovery
		\item Content reorganization
		\item Collaborative nature
		\item Multiple paths for problem solutions
		\item Intelligent feedback
		\item Personalized learning environment

	    \end{itemize}
\end{frame}

\begin{frame}
\frametitle{MOOCs: Types and Relation with classroom teaching}
 	\begin{figure}[ht]
	    \begin{center}
		%\includegraphics[scale=0.4]{moocs.jpg}
	    \end{center}
	    \caption{MOOCs: Types and relation}
	\end{figure}
\end{frame}


\section{Design perspectives}
\begin{frame}
	\frametitle{Foundation Stones for design}
	\begin{figure}[ht]
	    \begin{center}
		%\includegraphics[scale=0.4]{fs.jpg}
	    \end{center}
	   %\caption{Ref: Assembling pieces of MOOCs jigsaw puzzle- Sivamuni et al.}
	\end{figure} 
    Ref: Assembling pieces of MOOCs jigsaw puzzle- Sivamuni et al.
\end{frame}

\begin{frame}
	\frametitle{Design criteria}
	\begin{itemize}
	 \item Technological criteria
	    \begin{itemize}
		\item User interface
		\item Video content
		\item Learning \& social tools
		\item Learning analytics
	    \end{itemize}

	 \item Pedagogical criteria
	 \begin{itemize}
		\item Lecture organization
		\item Cultural diversity
		\item E-Assessment
		\item Peer assessment
	    \end{itemize}

	\end{itemize}

\end{frame}


\section{Evaluation techniques}
\begin{frame}
	\frametitle{Evaluation Schemes}
	\begin{itemize}
	  \item Instructor grading
	  \item Automated grading
	  \item Peer grading
	\end{itemize}
\end{frame}

\begin{frame}
	\frametitle{Automated grading vs peer grading}
	\begin{table}[ht]
	\resizebox{\textwidth}{!}{
	\begin{tabular}{|l|l|l|}
        \hline
        Criteria/Scheme           & \textbf{Automated Grading} & \textbf{Peer Grading}                 \\ \hline \hline
        Evaluator                 & Machine           & Course participant           \\ \hline
        Grading Scale             & Prefixed          & Dependant on submission type \\ \hline
        Score Bias                & Not biased        & Biased                       \\ \hline
        Variance from True Score  & Large             & Moderate                     \\ \hline
        Multiple Choice Questions & Supported         & Supported                    \\ \hline
        Short Answer              & Supported         & Supported                    \\ \hline
        Essay/Long Answer         & Not supported     & Supported                    \\ \hline
        Feedback                  & Not supported     & Supported(peer feedback)     \\ \hline
        Ground Truth Submissions  & Not required      & Supported                    \\ \hline
        Supported MOOC Types      & xMOOC             & cMOOC, xMOOC(small sized)    \\
        \hline
    \end{tabular}}
     \caption{Evaluation Techniques} \label{tab:eval}
\end{table}
\end{frame}

\begin{frame}
	\frametitle{Automatic Essay Scoring}
	\begin{itemize}
	  \item AES-Holistic grader
		\begin{itemize}
		 \item Overall score
		 \item Varies from true score
		 \item Effective in short answer type questions
		\end{itemize}

	  \item AES-Rubrics grader 
		\begin{itemize}
		    \item Multiple scores based on rubrics
		    \item Predecided rubrics
		    \item Final score is average
		    \item Effective in long answers/essay type questions
		    \end{itemize}
	   \item Limitations
	\end{itemize}

\end{frame}



\begin{frame}
	\frametitle{Peer grading models}
	\textbf{Aim} :Estimate the true score of peer graded submissions. \\
	\textbf{Terms}:\\  1] True score ($S_u$) for a submission $u$ : unknown/to be estimated.\\
	2] Grader bias ($b_v$) of grader $v$: tendency to inflate or deflate from actual score. \\ 
	3] Grader reliability($\tau_v$): the closeness of bias-corrected score to true score \\ 
	4] Observed score : median of set ($Z = {z}^{u}_{v}$) of scores assigned by all peer graders.\\
	Objective = $P(\{S_u\}_{u\in U}, \{b_v\}_{v \in V}, \{\tau_v\}_{v \in V }| Z)$ \newline
	Circular dependence: $S_u$  and $b_v$, posterior probability: \textbf{hard to compute}. \\
	\textbf{Gibbs sampling} to sample true score; average:- sampled true score $\hat{S_u}$. 
	%From this sampled true score, the grader biases are estimated, and in turn used to compute actual true scores.
\end{frame}

\section{Instructional quality}
\begin{frame}
	\frametitle{First principles of instruction}
		\begin{figure}[ht]
	    \begin{center}
		%\includegraphics[scale=0.25]{fpi.jpg}
	    \end{center}
	\end{figure}
	Ref: Instructional quality of MOOCs - Margaryan et al.
\end{frame}

\begin{frame}
	\frametitle{Myths about MOOCs}
	\begin{itemize}
	  \item MOOCs will affect instructional quality, faculties and TAs will be fired
	  \item MOOCs does not support small-group discussions, face-to-face interaction with instructor
	  \item MOOCs distract faculties from improvisation in on-campus pedagogy
	  \item MOOCs reduce diversity of teaching methods
	\end{itemize}
\end{frame}

\section{MOOCs potentials}
\begin{frame}
	\frametitle{Potential research areas}
	\begin{itemize}
	  \item Educational Data mining \& Learning analysis
	  \item Social network analysis methods
	  \item E-portfolio
	  \item Competence management
	  \item Self-review
	  \item Lifelong technical support
	  \item Content personalization
	  \item Certification
	\end{itemize}
\end{frame}


\section{Bodhitree}
\begin{frame}
\frametitle{Overview}
    \begin{itemize}
     \item Bodhitree is a platform being developed in CSE Department, IIT Bombay.
     \item Aim: to provide quality education through personalized, flexible, complete learning.
     \item Encourages blended MOOC concept.
     \item Course format: similar to xMOOCs.
     \item Instructional videos with embedded quizzes
     \item Student progress is recorded. 
     \item Course materials: form of chapters
     \item Discussion forums and chat-rooms
    \end{itemize}
\end{frame}
\begin{frame}
\frametitle{Architecture}    
\begin{figure}[ht]
\begin{center}
%\includegraphics[scale=0.25]{arch.jpg}
\end{center}
\caption{Tiered Architecture}\label{fig:arch}
\end{figure}
\end{frame}

\section{Conclusion \& Future work}
\begin{frame}
	\frametitle{Conclusion}
	\begin{itemize}
	  \item Selection of \textit{platform} is non-trivial.
	  \item Idea design specifications to be followed strictly. Evaluation techniques to be improved.
	  \item Learning analytics \& effective feedbacks are necessary.
	  \item Social network tools for building network.
	  \item Grading schemes can be enhanced using auto graders, intelligent peer grading.
	  \item Personalized learning environment creation, need for authoring tools, certification 
	\end{itemize}
\end{frame}


\begin{frame}
%	\frametitle{Thank you}
	\begin{center}
	      \textbf{Thank You}
	\end{center}

\end{frame}

\section{References}
\begin{frame}[shrink=50]
%\lipsum[10]
\frametitle{References}
%\setbeamerfont{References}{size=\small}
\begin{thebibliography}{1}
\bibitem{fox}Armando Fox. From MOOCs to SPOCs. In \textit{Communications of the ACM}, CASM’13, pages 38–40. ACM, 2013.

\bibitem{yousef1}Yousef, Chatti, Schroeder, Wosnitza. What Drives a Successful MOOC? An Empirical Examination of Criteria to Assure Design Quality of MOOCs. In \textit{Proceedings of 14th International Conference on Advanced Learning Technologies}, ICALT’14, pages 44–48. IEEE, 2014.

\bibitem{sivamuni} Sivamuni, Bhattacharya . Assembling pieces of MOOCs Jigsaw Puzzle. In \textit{Proceedings of International Conference in MOOC, Innovation and Technology in Education}, MITE’13, pages 393–398. IEEE, 2013.

\bibitem{kay}Kay, Reinmann, Diebold, Kummerfield . MOOCs:So Many Learners, So Much Potential ... In \textit{IEEE Intelligent Systems}, AIED’13, pages 2–9. IEEE, 2013.

\bibitem{sonwalkar} Nishikant Sonwalkar. The first Adaptive MOOC: A Case Study on Pedagogy Framework and Scalable Cloud Architecture-Part I, 2013.

\bibitem{reily}Reilly, Stafford, Williams, Corliss . Evaluating the Validity and Applicability of Automated Essay Scoring in two Massive Open Online Courses. In \textit{Proceedings of The International Review of
Research in Open and Distributed Learning}, IRRODL’14, 2014.

\bibitem{piech} Piech, Chen, Chuong Do, Huang, Andrew Ng, Koller. Tuned Models of Peer Assessment in MOOCs. In \textit{Communications of the ACM}, 2013.

\bibitem{yousef2} Yousef, Chatti, Schroeder, Wosnitza, Jakobs . Moocs: A Review of the State-of-the-Art. In \textit{Proceedings of 6th International Conference on Computer Supported Education}, csedu’14,pages 9–20. IEEE, 2014.

\bibitem{margaryan}  Margaryan, Bianco, Littlejohn. Instructional quality of massive open online courses(MOOCs).In \textit{Computers \& Education}, pages 77–83. Elsevier, 2014.

\bibitem {elearn} \href{http://www.webopedia.com/TERM/E/e_learning.html}{E-Learning information on Webopedia}

\bibitem {oer} \href{http://en.wikipedia.org/wiki/Open_educational_resources}{Open educational resources}

\bibitem {blendedmooc} \href{http://campustechnology.com/articles/2013/08/21/blended-moocs-the-best-of-both-worlds.aspx}{Blended MOOCs}

\bibitem {btree} \href{http://bodhitree.cse.iitb.ac.in/mission/}{ Bodhitree}

\end{thebibliography}

\end{frame}
%\end{thebibliography}
\end{document}

